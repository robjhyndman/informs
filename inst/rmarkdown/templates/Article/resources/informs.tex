%%%%%%%%%%%%%%%%%%%%%%%%%%%%%%%%%%%%%%%%%%%%%%%%%%%%%%%%%%%%%%%%%%%%%%%%%%%%
%% Author template for INFORMS journals
%% Mirko Janc, Ph.D., INFORMS, mirko.janc@informs.org
%% ver. 0.95, December 2010
%% Modified for Rmarkdown by Rob Hyndman. 11 April 2018
%%%%%%%%%%%%%%%%%%%%%%%%%%%%%%%%%%%%%%%%%%%%%%%%%%%%%%%%%%%%%%%%%%%%%%%%%%%%

\documentclass[$journal$,$if(blind)$blindrev$else$nonblindrev$endif$]{informs3}

%%\OneAndAHalfSpacedXI % current default line spacing
\OneAndAHalfSpacedXII
%%\DoubleSpacedXII
%%\DoubleSpacedXI

% If hyperref is used, dvi-to-ps driver of choice must be declared as
%   an additional option to the \documentclass. For example
%\documentclass[dvips,mksc]{informs3}      % if dvips is used
%\documentclass[dvipsone,mksc]{informs3}   % if dvipsone is used, etc.

% Natbib setup for author-year style
\usepackage{natbib}
 \bibpunct[, ]{(}{)}{,}{a}{}{,}%
 \def\bibfont{\small}%
 \def\bibsep{\smallskipamount}%
 \def\bibhang{24pt}%
 \def\newblock{\ }%
 \def\BIBand{and}%


%% Setup of theorem styles. Outcomment only one.
%% Preferred default is the first option.
\TheoremsNumberedThrough     % Preferred (Theorem 1, Lemma 1, Theorem 2)
%\TheoremsNumberedByChapter  % (Theorem 1.1, Lema 1.1, Theorem 1.2)

%% Setup of the equation numbering system. Outcomment only one.
%% Preferred default is the first option.
\EquationsNumberedThrough    % Default: (1), (2), ...
%\EquationsNumberedBySection % (1.1), (1.2), ...

% In the reviewing and copyediting stage enter the manuscript number.
$if(manuscriptno)$\MANUSCRIPTNO{$manuscriptno$}$endif$
%\MANUSCRIPTNO{} % When the article is logged in and DOI assigned to it,
                 %   this manuscript number is no longer necessary

\usepackage{ifxetex,ifluatex}
\usepackage{fixltx2e} % provides \textsubscript

\usepackage[T1]{fontenc}
\usepackage[utf8]{inputenc}

% use upquote if available, for straight quotes in verbatim environments
\IfFileExists{upquote.sty}{\usepackage{upquote}}{}
\PassOptionsToPackage{hyphens}{url} % url is loaded by hyperref
$if(verbatim-in-note)$
\usepackage{fancyvrb}
$endif$
\usepackage[unicode=true]{hyperref}
$if(colorlinks)$
\PassOptionsToPackage{usenames,dvipsnames}{color} % color is loaded by hyperref
$endif$
\hypersetup{
$if(title-meta)$
            pdftitle={$title-meta$},
$endif$
$if(author-meta)$
            pdfauthor={$author-meta$},
$endif$
$if(keywords)$
            pdfkeywords={$for(keywords)$$keywords$$sep$, $endfor$},
$endif$
$if(colorlinks)$
            colorlinks=true,
            linkcolor=$if(linkcolor)$$linkcolor$$else$Maroon$endif$,
            citecolor=$if(citecolor)$$citecolor$$else$Blue$endif$,
            urlcolor=$if(urlcolor)$$urlcolor$$else$Blue$endif$,
$else$
            pdfborder={0 0 0},
$endif$
            breaklinks=true}
\urlstyle{same}  % don't use monospace font for urls
$if(verbatim-in-note)$
\VerbatimFootnotes % allows verbatim text in footnotes
$endif$
$if(geometry)$
\usepackage[$for(geometry)$$geometry$$sep$,$endfor$]{geometry}
$else$
\usepackage{geometry}
\geometry{a4paper, text={16cm,24cm}}
$endif$
$if(lang)$
\ifnum 0\ifxetex 1\fi\ifluatex 1\fi=0 % if pdftex
  \usepackage[shorthands=off,$for(babel-otherlangs)$$babel-otherlangs$,$endfor$main=$babel-lang$]{babel}
$if(babel-newcommands)$
  $babel-newcommands$
$endif$
\else
  \usepackage{polyglossia}
  \setmainlanguage[$polyglossia-lang.options$]{$polyglossia-lang.name$}
$for(polyglossia-otherlangs)$
  \setotherlanguage[$polyglossia-otherlangs.options$]{$polyglossia-otherlangs.name$}
$endfor$
\fi
$endif$
$if(natbib)$
\usepackage{natbib}
%\bibliographystyle{$if(biblio-style)$$biblio-style$$else$plainnat$endif$}
$endif$
$if(biblatex)$
\usepackage[$if(biblio-style)$style=$biblio-style$,$endif$$for(biblatexoptions)$$biblatexoptions$$sep$,$endfor$]{biblatex}
$for(bibliography)$
\addbibresource{$bibliography$}
$endfor$
$endif$
$if(listings)$
\usepackage{listings}
$endif$
$if(lhs)$
\lstnewenvironment{code}{\lstset{language=Haskell,basicstyle=\small\ttfamily}}{}
$endif$
$if(highlighting-macros)$
$highlighting-macros$
$endif$
$if(tables)$
\usepackage{longtable,booktabs}
% Fix footnotes in tables (requires footnote package)
\IfFileExists{footnote.sty}{\usepackage{footnote}\makesavenoteenv{long table}}{}
$endif$
\makeatletter
\def\maxwidth{\ifdim\Gin@nat@width>\linewidth\linewidth\else\Gin@nat@width\fi}
\def\maxheight{\ifdim\Gin@nat@height>\textheight\textheight\else\Gin@nat@height\fi}
\makeatother
% Scale images if necessary, so that they will not overflow the page
% margins by default, and it is still possible to overwrite the defaults
% using explicit options in \includegraphics[width, height, ...]{}
\setkeys{Gin}{width=\maxwidth,height=\maxheight,keepaspectratio}
$if(links-as-notes)$
% Make links footnotes instead of hotlinks:
\renewcommand{\href}[2]{#2\footnote{\url{#1}}}
$endif$
$if(strikeout)$
\usepackage[normalem]{ulem}
% avoid problems with \sout in headers with hyperref:
\pdfstringdefDisableCommands{\renewcommand{\sout}{}}
$endif$
$if(indent)$
$else$
\IfFileExists{parskip.sty}{%
\usepackage{parskip}
}{% else
\setlength{\parindent}{0pt}
\setlength{\parskip}{6pt plus 2pt minus 1pt}
}
$endif$
\setlength{\emergencystretch}{3em}  % prevent overfull lines
\providecommand{\tightlist}{%
  \setlength{\itemsep}{0pt}\setlength{\parskip}{0pt}}
$if(numbersections)$
\setcounter{secnumdepth}{$if(secnumdepth)$$secnumdepth$$else$5$endif$}
$else$
\setcounter{secnumdepth}{0}
$endif$
$if(dir)$
\ifxetex
  % load bidi as late as possible as it modifies e.g. graphicx
  $if(latex-dir-rtl)$
  \usepackage[RTLdocument]{bidi}
  $else$
  \usepackage{bidi}
  $endif$
\fi
\ifnum 0\ifxetex 1\fi\ifluatex 1\fi=0 % if pdftex
  \TeXXeTstate=1
  \newcommand{\RL}[1]{\beginR #1\endR}
  \newcommand{\LR}[1]{\beginL #1\endL}
  \newenvironment{RTL}{\beginR}{\endR}
  \newenvironment{LTR}{\beginL}{\endL}
\fi
$endif$

% set default figure placement to htbp
\makeatletter
\def\fps@figure{htbp}
\makeatother


$if(title)$
\title{$title$}
$endif$
$if(subtitle)$
\providecommand{\subtitle}[1]{}
\subtitle{$subtitle$}
$endif$
$if(institute)$
\providecommand{\institute}[1]{}
\institute{$for(institute)$$institute$$sep$ \and $endfor$}
$endif$


%% GRAPHICS
\RequirePackage{graphicx}
\setcounter{topnumber}{2}
\setcounter{bottomnumber}{2}
\setcounter{totalnumber}{4}
\renewcommand{\topfraction}{0.85}
\renewcommand{\bottomfraction}{0.85}
\renewcommand{\textfraction}{0.15}
\renewcommand{\floatpagefraction}{0.8}

%% LINE AND PAGE BREAKING
\sloppy
\clubpenalty = 10000
\widowpenalty = 10000
\brokenpenalty = 10000

\usepackage[showonlyrefs]{mathtools}
\usepackage[no-weekday]{eukdate}

% Need to prevent theorem and amsthm clashing
\makeatletter
\let\proof\relax
\let\endproof\relax
\let\theoremstyle\relax
\let\theorem\relax
\let\endtheorem\relax
\let\c@theorem\relax
\let\lemma\relax
\let\endlemma\relax
\let\c@lemma\relax
\let\definition\relax
\let\enddefinition\relax
\let\c@definition\relax
\let\corollary\relax
\let\endcorollary\relax
\let\c@corollary\relax
\let\proposition\relax
\let\endproposition\relax
\let\c@proposition\relax
\let\example\relax
\let\endexample\relax
\let\c@example\relax
\let\exercise\relax
\let\endexercise\relax
\let\c@exercise\relax
\let\remark\relax
\let\endremark\relax
\let\c@remark\relax
\let\conjecture\relax
\let\endconjecture\relax
\let\c@conjecture\relax
\makeatother

$for(header-includes)$
$header-includes$
$endfor$

\begin{document}

% Outcomment only when entries are known. Otherwise leave as is and
%   default values will be used.
%\setcounter{page}{1}
%\VOLUME{00}%
%\NO{0}%
%\MONTH{Xxxxx}% (month or a similar seasonal id)
%\YEAR{0000}% e.g., 2005
%\FIRSTPAGE{000}%
%\LASTPAGE{000}%
%\SHORTYEAR{00}% shortened year (two-digit)
%\ISSUE{0000} %
%\LONGFIRSTPAGE{0001} %
%\DOI{10.1287/xxxx.0000.0000}%

% Author's names for the running heads
% Sample depending on the number of authors;
\RUNAUTHOR{$for(author)$$author.familyname$$sep$, $endfor$}
% \RUNAUTHOR{Jones}
% \RUNAUTHOR{Jones and Wilson}
% \RUNAUTHOR{Jones, Miller, and Wilson}
% \RUNAUTHOR{Jones et al.} % for four or more authors
% Enter authors following the given pattern:
%\RUNAUTHOR{}

% Title or shortened title suitable for running heads. Sample:
% \RUNTITLE{Bundling Information Goods of Decreasing Value}
% Enter the (shortened) title:
$if(runningtitle)$\RUNTITLE{$runningtitle$}$else$\RUNTITLE{$title$}$endif$

% Full title. Sample:
% \TITLE{Bundling Information Goods of Decreasing Value}
% Enter the full title:
\TITLE{$title$}

% Block of authors and their affiliations starts here:
% NOTE: Authors with same affiliation, if the order of authors allows,
%   should be entered in ONE field, separated by a comma.
%   \EMAIL field can be repeated if more than one author
\ARTICLEAUTHORS{%
$for(author)$
\AUTHOR{$author.othernames$~$author.familyname$}
\AFF{$author.affiliation$$if(author.email)$. \EMAIL{$author.email$}$endif$$if(author.url)$ \URL{$author.url$}$endif$}
$endfor$
} % end of the block

$if(abstract)$
\ABSTRACT{$abstract$}
$endif$
$if(keywords)$
\KEYWORDS{$keywords$}
$endif$
$if(title)$
\maketitle
$endif$

$for(include-before)$
$include-before$
$endfor$

% Samples of sectioning (and labeling) in MKSC
% NOTE: (1) \section and \subsection do NOT end with a period
%       (2) \subsubsection and lower need end punctuation
%       (3) capitalization is as shown (title style).
%
%\section{Introduction.}\label{intro} %%1.
%\subsection{Duality and the Classical EOQ Problem.}\label{class-EOQ} %% 1.1.
%\subsection{Outline.}\label{outline1} %% 1.2.
%\subsubsection{Cyclic Schedules for the General Deterministic SMDP.}
%  \label{cyclic-schedules} %% 1.2.1
%\section{Problem Description.}\label{problemdescription} %% 2.

% Text of your paper here


$body$

% Acknowledgments here
\ACKNOWLEDGMENT{%
$acknowledgment$
}% Leave this (end of acknowledgment)


% Appendix here
% Options are (1) APPENDIX (with or without general title) or
%             (2) APPENDICES (if it has more than one unrelated sections)
% Outcomment the appropriate case if necessary
%
% \begin{APPENDIX}{<Title of the Appendix>}
% \end{APPENDIX}
%
%   or
%
% \begin{APPENDICES}
% \section{<Title of Section A>}
% \section{<Title of Section B>}
% etc
% \end{APPENDICES}

% References here (outcomment the appropriate case)

% CASE 1: BiBTeX used to constantly update the references
%   (while the paper is being written).
\bibliographystyle{informs2014} % outcomment this and next line in Case 1
$if(bibliography)$
\bibliography{$for(bibliography)$$bibliography$$sep$,$endfor$}
$endif$

%\bibliography{<your bib file(s)>} % if more than one, comma separated

% CASE 2: BiBTeX used to generate mypaper.bbl (to be further fine tuned)
%\input{mypaper.bbl} % outcomment this line in Case 2


$for(include-after)$
$include-after$
$endfor$

\end{document}
